\phantomsection
\setsection{Chương 5: Kết luận và hướng phát triển}
\setcounter{section}{5}
\setcounter{subsection}{0}

\phantomsection

\subsection{Kết quả đạt được}
Về mặt thực tiễn: Đồ án đã giải quyết thách thức ban đầu. Đề tài "Website cửa hàng sách" đã được thiết kế với các chức năng linh hoạt để đáp ứng nhu cầu sử dụng của người dùng. Sử dụng công nghệ Mongoose để quản lý dữ liệu, Bootstrap 5 và Ant Design để tạo giao diện thân thiện và hấp dẫn hơn, kết hợp JWT để bảo vệ tính xác thực của người dùng, cùng với các công nghệ khác để tạo ra một trang web hoàn chỉnh.
\par
Các chức năng cơ bản như tìm kiếm sách, mượn/trả sách, và quản lý tài khoản đã được triển khai một cách hoàn chỉnh và linh hoạt, đáp ứng đủ nhu cầu của người dùng.

\subsection{Hạn chế}
Tuy nhiên, có những hạn chế nhất định do thời gian thực hiện dự án hạn chế, không cho phép việc tối ưu hóa và tùy chỉnh cao hơn như thêm chức năng quên mật khẩu và điều chỉnh giao diện theo sở thích cá nhân của người dùng.


\phantomsection

\subsection{Hướng phát triển}
Để tối ưu hóa trải nghiệm của người dùng và tăng tính hiệu quả của trang web, có một số hướng phát triển tiềm năng như sau:

\begin{itemize}[label={+}]
          \item Phân loại sách theo thể loại và chủ đề để giúp người dùng dễ dàng tìm kiếm và chọn lọc sách hơn.
          \item Phân loại sách: Thêm tính năng phân loại sách theo thể loại trong cơ sở dữ liệu, giúp người dùng tìm kiếm và lựa chọn sách dễ dàng hơn theo thể loại và giá tiền.
          \item Tùy chỉnh giao diện: Cung cấp khả năng tùy chỉnh giao diện cho người dùng, bao gồm chế độ sáng/tối và đa ngôn ngữ, nhằm tối ưu hóa trải nghiệm người dùng.
\end{itemize}