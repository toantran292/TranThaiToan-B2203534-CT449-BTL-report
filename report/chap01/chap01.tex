\phantomsection
\setsection{Chương 1: Giới thiệu}
\setcounter{section}{1}

\phantomsection
\subsection{Đặt vấn đề}
Trong xã hội ngày nay, sự phát triển của công nghệ đã mở ra một cánh cửa rộng lớn cho thương mại điện tử, với các website bán hàng trở thành một phần không thể thiếu của cuộc sống hàng ngày. Các trang web này không chỉ đơn giản là nơi mua sắm, mà còn đóng vai trò quan trọng trong việc tạo ra sự thuận tiện và tiết kiệm thời gian cho người tiêu dùng. Các website bán hàng cung cấp sự đa dạng về sản phẩm và dịch vụ, từ hàng hóa tiêu dùng đến sản phẩm công nghệ, từ quần áo đến đồ gia dụng. Sự đa dạng này giúp người tiêu dùng có thể dễ dàng tìm kiếm và so sánh giữa các sản phẩm để chọn lựa những điều phù hợp nhất với nhu cầu và ngân sách của mình.\par
Với cuộc sống hiện đại ngày càng bận rộn, việc có thể mua hàng trực tuyến từ bất kỳ đâu và bất kỳ khi nào là điều cực kỳ quan trọng. Các website bán hàng mang lại sự thuận tiện và linh hoạt cho người tiêu dùng, giúp họ tiết kiệm thời gian và công sức trong quá trình mua sắm.\par
Ngoài ra, một yếu tố quan trọng khác trong xã hội hiện đại là việc chia sẻ kiến thức và thông tin. Đó là lý do tại sao một website có khả năng bán sách cho mượn sách trở nên cần thiết. Việc này không chỉ đáp ứng nhu cầu đọc sách của mọi người mà còn tạo ra cơ hội cho việc chia sẻ tri thức và kinh nghiệm. Bằng cách cung cấp sách cho mượn, một website không chỉ là một điểm đến để tiếp cận kiến thức mà còn là một nền tảng cho sự trao đổi ý kiến và tương tác văn hóa.\par
Nhìn chung, các website bán hàng và những nền tảng bán sách cho mượn không chỉ là các công cụ thương mại, mà còn là các phương tiện quan trọng trong việc tạo ra sự tiện lợi, đa dạng và chia sẻ kiến thức trong xã hội ngày nay.


\phantomsection
\subsection{Mục tiêu đề tài}

\phantomsection
\subsection{Đối tượng và phạm vi nghiên cứu}

\phantomsection
\subsection{Phương pháp nghiên cứu}

\phantomsection
\subsection{Các chức năng chính}
Website quản lý cho mượn sách được tạo ra nhằm giúp người quản lý dễ dàng trong việc quản lý việc cho mượn sách, tra cứu được thời gian cho mượn và thời gian trả của từng cuốn sách. \par
Đối với người mượn sách có thể dễ dàng tìm kiếm các cuốn sách hiện có và đặt mượn một cách trực tuyến.

\begin{itemize}[align=left, leftmargin=2cm]
  \item[\textbf{-- Đối với quản trị viên}]:
        \begin{itemize}[label={+}]
          \item Quản lý người dùng (tạo tài và cấp quyền quản trị khoản người dùng, vô hiệu quá tài khoản)
          \item Quản lý sách (Số lượng sách hiện còn, lịch sử được mượn, thêm ảnh cho sách)
          \item Quản lý tác giả
          \item Quản lý nhà xuất bản
        \end{itemize}
  \item[\textbf{-- Đối với khách hàng}]:
        \begin{itemize}[label={+}]
          \item Quản lý người dùng (tạo tài và cấp quyền quản trị khoản người dùng, vô hiệu quá tài khoản)
          \item Quản lý sách (Số lượng sách hiện còn, lịch sử được mượn, thêm ảnh cho sách)
          \item Quản lý tác giả
          \item Quản lý nhà xuất bản
        \end{itemize}
\end{itemize}
