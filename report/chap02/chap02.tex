\phantomsection
\setsection{Chương 2: Đặc tả yêu cầu}
\setcounter{section}{2}
\setcounter{subsection}{0}

\phantomsection

\subsection{Mô tả bài toán}
Thiết kế website quản lý mượn sách để đáp ứng nhu cầu mượn sách của người dùng một cách dễ dàng, ít tốn thời gian phải ra trực tiếp nơi cho mượn sách để tìm xem cuốn sách mình cần có còn hay không. Người dùng muốn mượn sách chỉ cần truy cập vào website và chọn những quyển sách mình muốn mượn sau đó ra nơi cho mượn để nhận sách là hoàn thành. Về phía người quản trị có thể xem được lịch sử mượn của một cuốn sách nào đó, thêm hoặc bớt đi số lượng cho các cuốn sách, thay đổi thông tin của các cuốn sách, ...
\par
Với khách vãng lai thì có thể vào trang web để xem tình trạng các quyển sách (có hay không, còn dư để mượn hay không) cũng như thông tin chi tiết về các quyển sách đó nhưng không thể đặt mượn trực tuyến mà phải ra nơi cho mượn sách để mượn (có thể lúc tìm kiếm thì sách còn nhưng khi ra đến nơi sách đã bị người khác mượn trước). Để có thể đặt mượn sách trực tuyến người dùng phải đăng ký thành viên của trang web. Khi đã là thành viên của trang web, người dùng đăng nhập vào website theo tên đăng nhập và mật khẩu của mình, sau khi đăng nhập thì người dùng có quyền đặt mượn trực tuyến.

\phantomsection

\subsection{Yêu cầu bài toán}

\begin{itemize}[align=left, leftmargin=2cm]
  \item[\textbf{-- Đối với khách vãng lai (Guest)}]:
        \begin{itemize}[label={+}]
          \item Được xem thông tin các quyển sách, tác giả sách và nhà xuất bản sách.
          \item Tìm kiếm các quyển sách phù hợp
        \end{itemize}
  \item[\textbf{-- Đối với khách hàng (User)}]:
        \begin{itemize}[label={+}]
          \item Gồm các chức năng của khách vãng lai
          \item Được quyền đặt mượn sách trực tuyến.
          \item Tra cứu thời hạn trả sách.
          \item Tra cứu lịch sử mượn sách.
        \end{itemize}
  \item[\textbf{-- Đối với quản trị viên (User)}]:
        \begin{itemize}[label={+}]
          \item Thêm bớt số lượng quyển sách, ẩn sách khỏi trang web.
          \item Quản lý người dùng, vô hiệu quá tài khoản vi phạm.
          \item Quản lý nhà xuất bản, tác giả
        \end{itemize}
\end{itemize}


\phantomsection
\subsection{Ngôn ngữ lập trình, thư viện và các công cụ có liên quan}
% \setcounter{subsection}

\phantomsection
\subsubsection{Vue}
\begin{center}
  \begin{minipage}{.3\linewidth}
    \captionsetup{type=figure, width=.93\linewidth}
    \includegraphics[width=\linewidth]{images/vue.png}
    \caption{\centering Vue}
    \label{fig:vue}
  \end{minipage}%
\end{center}

Vue.js (Vue) là một framework JavaScript mã nguồn mở, được sử dụng để xây dựng giao diện người dùng động và tương tác.
Vue có cú pháp dễ đọc, hỗ trợ hai chiều dữ liệu và tích hợp tốt với các thư viện khác \cite{vue}.

\phantomsection
\subsubsection{Node}
\figmini[.5]{images/node.png}{fig:node}{NodeJS}

Node.js (Node) là một môi trường chạy mã JavaScript phía máy chủ.
Nó cho phép viết mã JavaScript không chỉ trong trình duyệt mà còn trên máy chủ, giúp xây dựng ứng dụng web đa năng \cite{node}.

\phantomsection
\subsubsection{Express}
\figmini{images/express.png}{fig:express}{ExpressJS}

Express là một framework Node.js giúp xây dựng các ứng dụng web và API một cách nhanh chóng.
Nó cung cấp các tiện ích để quản lý định tuyến, xử lý yêu cầu và phản hồi \cite{express}.

\phantomsection
\subsubsection{Mongo}
\figmini{images/mongo.png}{fig:mongo}{MongoDB}

MongoDB là một hệ quản trị cơ sở dữ liệu phi quan hệ, dựa trên tài liệu.
Nó lưu trữ dữ liệu dưới dạng JSON-like và hỗ trợ mở rộng dễ dàng \cite{mongodb}.

\phantomsection
\subsubsection{Ant Design Vue}
\figmini{images/antdv.png}{fig:antdv}{Ant Design Vue}

Ant Design Vue là một framework dựa trên Vue.js, được thiết kế theo nguyên tắc Material Design.
Nó cung cấp các thành phần và công cụ giúp bạn xây dựng giao diện người dùng đẹp và phong phú về nội dung \cite{antdesignvue}.
\phantomsection
\subsubsection{Vuetify}
\figmini{images/vuetify.png}{fig:vuetify}{Vuetify}

Vuetify là một framework dựa trên Vue.js với thiết kế theo nguyên tắc Material Design.
Nó cung cấp các thành phần và công cụ giúp bạn xây dựng ứng dụng đẹp và phong phú về nội dung \cite{vuetify}.

\phantomsection
\subsubsection{Visual Studio Code}
\figmini{images/vscode.png}{fig:vscode}{Visual Studio Code}

Visual Studio Code (VS Code) là một trình soạn thảo mã nguồn nhẹ nhưng mạnh mẽ, chạy trên máy tính của bạn và có sẵn cho Windows, macOS và Linux.
Nó tích hợp sẵn hỗ trợ cho JavaScript, TypeScript và Node.js, và có một hệ sinh thái phong phú về tiện ích mở rộng cho các ngôn ngữ và môi trường khác (như C++, C\#, Java, Python, PHP, Go, .NET) \cite{vscode}.




